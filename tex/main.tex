\documentclass[9pt,twoside]{extarticle}
\usepackage{style}

\begin{document}
  \begin{multicols*}{3}
    \tableofcontents

    \section{Struktury danych}
    \ctitle{Drzewo falkowe}
    \cheader{Opis}{%
      Obsługuje zapytania typu \textit{podaj k-ty najmniejszy na przedziale} itp.~na statycznej tablicy.
      Jeżeli czas albo pamięć są ciasne warto przeskalować liczby.
      Niszczy tablicę.
    }
    \cheader{Czas}{$\bigo(\log A)$}
    \cfile{../data/wavelet.cpp}

    \ctitle{Ordered set}
    \cheader{Opis}{Alternatywnie można użyć treapa albo trie.}
    \cheader{Stosowanie}{\src{s.find_by_order(k)} i \src{s.order_of_key(k)}.}
    \cheader{Czas}{$\bigo(\log n)$ z dużą stałą.}
    \cfile{../data/pbds.cpp}

    \section{Matma}
    \subsection{Mnożniki Lagrange'a}
    Jeżeli optymalizujemy $f(x_1, \dots, x_n)$, pod warunkami typu $g_k(x_1, \dots, x_n) = 0$ to
    $x_1, \dots, x_n$ jest ekstremum lokalnym tylko jeżeli gradient $\nabla f(x_1, \dots, x_n)$ jest kombinacją liniową
    gradientów $\nabla g_k(x_1, \dots x_n)$.
  \end{multicols*}
\end{document}
